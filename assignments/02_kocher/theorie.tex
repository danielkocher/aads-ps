\documentclass{article}
\usepackage{graphicx}
\usepackage[utf8]{inputenc}
\usepackage[linesnumbered]{algorithm2e}
\usepackage[margin=.75in]{geometry}
\newenvironment{algorithmic}{%
\renewenvironment{algocf}[1][h]{}{}% pass over the floating stuff
\algorithm
}{%
\endalgorithm
}
\begin{document}
%
\title{Assignment 2 \\ Advanced Algorithms \& Data Structures PS}%
\author{Christian Müller 1123410 \\ Daniel Kocher, 0926293}%
\maketitle%
\clearpage%
%
{\noindent\bfseries Aufgabe 4}%

\medskip%

{\noindent\LARGE VORSCHLAG KOCHER}%

\medskip%
\noindent
F{\"u}r den Divide-\&-Conquer Algorithmus zur Berechnung von Segmentschnittpunkten
werden zuerst zwei Datenstrukturen definiert, welche verwendet werden um die
Elemente der Liste $S$ darzustellen.

\paragraph{Endpunkt}
Stellt einen Endpunkt eines horizontalen Segments dar.
Jeder Endpunkt besitzt mindestens drei Felder:
\begin{itemize}
  \item[-] $x$ \ldots repr{\"a}sentiert die x-Koordinate des Endpunktes.
  \item[-] $y$ \ldots repr{\"a}sentiert die y-Koordinate des Endpunktes.
  \item[-] $ass$ \ldots Pointer zum zugeh{\"o}rigen Partner-Endpunkt.
\end{itemize}

\emph{Beispiel}: Das horizontale Segment mit den Endpunkten $(1, 2)$ und $(4, 2)$ wird
durch zwei {\bfseries Endpunkte} $e_{1, 2}$ repr{\"a}sentiert. Die Felder von
$e_{1, 2}$ sind wie folgt belegt:

\begin{center}
\begin{tabular}{l||l|l|l|}
  \hline
  $e_1$ & $.x = 1$ & $.y = 2$ & $.ass = e_2$ \tabularnewline
  \hline
  $e_2$ & $.x = 4$ & $.y = 2$ & $.ass = e_1$ \tabularnewline
  \hline
\end{tabular}
\end{center}

\paragraph{VertSegment}
Stellt ein vertikales Segment dar. Jedes vertikale Segment besitzt mindestens
drei Felder:
\begin{itemize}
  \item[-] $x$ \ldots repr{\"a}sentiert die x-Koordinate des vertikalen Segments.
  \item[-] $y_{unten}$ \ldots repr{\"a}sentiert die untere y-Koordinate des
    vertikalen Segments.
  \item[-] $y_{oben}$ \ldots repr{\"a}sentiert die obere y-Koordinate des
    vertikalen Segments.
\end{itemize}

\emph{Beispiel}: Das vertikale Segment mit den Endpunkten $(1, 6)$ und $(1, 1)$
wird durch ein {\bfseries VertSegment} $v$ repr{\"a}sentiert. Die Felder von $v$
sind wie folgt belegt:

\begin{center}
\begin{tabular}{l||l|l|l|}
  \hline
  $v$ & $.x = 1$ & $.y_{unten} = 1$ & $.y_{oben} = 6$ \tabularnewline
  \hline
\end{tabular}
\end{center}

%
Desweiteren werden f{\"u}r jede Liste $S_i$ jeweils drei Listen $L(S_i)$, $R(S_i)$
und $V(S_i)$ global verwaltet. Dies k{\"o}nnten auch lokale Listen sein,
allerdings geht nachfolgender Algorithmus von global verwalteten Listen aus.

$L(S_i)$ enth{\"a}lt die y-Koordinaten aller linken Endpunkte in $S_i$, deren
rechter Partner ($ass$) nicht in $S_i$ enthalten ist. Analog beinhaltet $R(S_i)$
die y-Koordinaten aller rechten Endpunkte in $S_i$, deren linker Partner ($ass$)
nicht in $S_i$ liegt. $V(S_i$) hingegen enth{\"a}lt die y-Intervalle ($y_{unten}$
bis $y_{oben}$) der vertikalen Segmente in $S_i$.

$L(S_i)$ und $R(S_i)$ sind nach steigenden y-Koordinaten sortierte, einfach
verkettete Listen. $V(S_i)$ sind nach steigenden unteren Endpunkten ($y_{unten}$)
sortierte, einfach verkettete Listen.

Seien $l$, $l_i$ einfach verkettete Listen und $d$ ein Wert gespeichert in einer
einfach verketteten Liste. Solche Listen haben mindestens folgende Felder:
\begin{itemize}
  \item[-] $head$ \ldots Erster Knoten einer Liste.
\end{itemize}

\noindent
Weiters stellen sie folgende Funktionen zur Verf{\"u}gung:
\begin{itemize}
  \item[-] $l_1$.deleteAll($l_2$) \ldots Entfernt alle Element von Liste $l_2$ aus
    Liste $l1$.
  \item[-] $l_1$.insertAll($l_2$) \ldots F{\"u}gt alle Element von Liste $l_2$ zu
    Liste $l_1$ hinzu.
  %\item[-] $l$.getHead() \ldots Gibt den ersten Knoten der Liste $l$ zur{\"u}ck.
  \item[-] $l$.search($d$) \ldots Durchsucht die Liste $l$ nach einem Element $d$
    und gibt dessen Knoten zur{\"u}ck (falls gefunden).
  \item[-] $l$.insert($d$) \ldots F{\"u}gt das Element $d$ zu Liste $l$ hinzu.
  \item[-] $l$.delete($d$) \ldots Entfernt das Element $d$ aus Liste $l$.
\end{itemize}
%
Knoten einer solchen Liste stellen mindestens die folgenden Felder bereit:
\begin{itemize}
  \item[-] $next$ \ldots N{\"a}chster Knoten.
  %\item[-] $data$ \ldots Korrespondierender Datensatz.
\end{itemize}
%
\begin{algorithm}
\begin{algorithmic}[1]
  \SetKwProg{Fn}{Function}{}{}%
  \SetKwInOut{Input}{Input}%
  \SetKwInOut{Output}{Output}%
%
  \Input{
    Liste $S$ bestehend aus vertikalen Segmenten ({\bfseries VertSegment}) und
    Endpunkten ({\bfseries Endpunkt}) von horizontalen Segmenten. \newline%
    $S$ sei nach horizontalen Koordinaten sortiert (Algorithmus ist leicht zu
    adaptieren falls nicht).
  }%
  \Output{
    Alle Schnittpunkte von vertikalen Segmenten mit horizontalen Segmenten
  }%
  \Fn{ReportCuts($S$)}{%
    \lIf{$|S| \leq 0$}{\Return}%
    \If{$|S| = 1$}{
      \tcp{Initialisierung von $L(S)$, $R(S)$ und $V(S)$}
      Sei $s$ das einzige Element in $S$\;
      \BlankLine%
      \lIf{$s$ ist linker Endpunkt}{$L(S) \leftarrow \{s\}$, $R(S) \leftarrow \emptyset$, $V(S) \leftarrow \emptyset$}%
      \BlankLine%
      \lElseIf{$s$ ist rechter Endpunkt}{$L(S) \leftarrow \emptyset$, $R(S) \leftarrow \{s\}$, $V(S) \leftarrow \emptyset$}% 
      \BlankLine%
      \lElseIf{$s$ ist vertikales Segment}{$L(S) \leftarrow \emptyset$, $R(S) \leftarrow \emptyset$, $V(S) \leftarrow \{s\}$}% 
      \BlankLine%
      \Return \tcp*{Rekursionsende}%
    }%
    \BlankLine%
    \lIf{$S$ ist nicht bez{\"u}glich x-Koordinaten sortiert}{Sortiere $S$ bez{\"u}glich x-Koordinaten}%
    \BlankLine%
    \tcc{
      Divide Schritt. \newline
      $m$ sei ein Integer (repr{\"a}sentiert die Mitte der Liste $S$). \newline
      $S_1$ und $S_2$ seien Listen von vertikalen Segmenten und Endpunkten von
      horizontalen Segmenten, sortiert bezüglich ihrer x-Koordinaten.
    }%
    \lIf{$|S|$ ist gerade}{$m \leftarrow |S| / 2$}%
    \lElse{$m \leftarrow \lceil |S| / 2 \rceil$}%
    $S_1 \leftarrow $ Liste der ersten $m$ Elemente aus $S$\;
    $S_2 \leftarrow $ Liste der letzten $\left( |S| - m \right)$ Elemente aus $S$\;
    \BlankLine%
    \tcp{Conquer Schritt}%
    ReportCuts($S_1$)\;%
    ReportCuts($S_2$)\;%
    \BlankLine%
    \tcc{
      $L(S_i)$, $R(S_i)$, $V(S_i)$ f{\"u}r $i = 1, 2$ bekannt $\Rightarrow$ Merge
      Schritt \newline%
      Berichte Segmentschnittpunkte (Paare $(h, v)$)%
    }%
    $h_1 \leftarrow R(S_2)$\;
    DeletePartnersOf($h_1$, $L(S_1)$)\;
    $v_1 \leftarrow V(S_1)$\;
    IntersectAndReport($h_1$, $v_1$)\;
    \BlankLine%
    $h_2 \leftarrow L(S_1)$\;
    DeletePartnersOf($h_2$, $R(S_2)$)\;
    $v_2 \leftarrow V(S_2)$\;
    IntersectAndReport($h_2$, $v_2$)\;
    
    \BlankLine%
    \tcp{Aktualisiere $L(S)$, $R(S)$ und $V(S)$ f{\"u}r $S = S_1 \cup S_2$}
    $L(S) \leftarrow L(S_1)$\;
    $L(S)$.DeleteAll($R(S_2)$)\;
    $L(S)$.InsertAll($L(S_2)$)\;
    \BlankLine%
    $R(S) \leftarrow R(S_2)$\;
    $R(S)$.DeleteAll($L(S_1)$)\;
    $R(S)$.InsertAll($R(S_1)$)\;
    \BlankLine%
    $V(S) \leftarrow V(S_1)$\;
    $V(S)$.InsertAll($V(S_2)$)\;
  }%
\end{algorithmic}
\end{algorithm}
%
\begin{algorithm}                     
\begin{algorithmic} [1]  
  \SetKwProg{Fn}{Function}{}{}%                 % enter the algorithmic environment
  \SetKwInOut{Input}{Input}%
  \SetKwInOut{Output}{Output}%
%
  \Input{
    Zwei einfach verkettete Listen $l_1$ und $l_2$. \newline%
    $l_{1,2}$ beinhalten nur Endpunkte von horizontalen Segmenten.
  }%
  \Output{
    Eine Liste, welche alle Elemente aus $l_1$ beinhaltet, die keinen Partner in
    $l_2$ besitzen.
  }%

  \Fn{DeletePartnersOf($l_1$, $l_2$)}{
    \lIf{$l_2$ ist leer}{\Return $l_1$}

    $current \leftarrow l_2.head$\;
    \While{$current \neq null$}{
      \lIf{$l_1.search(current.ass)$ hat einen Partner gefunden}{
        $l_1.delete(current)$
      }
      $current \leftarrow current.next$\;
    }

    \Return $l_1$\;
  }
\end{algorithmic}
\end{algorithm}
%
\begin{algorithm}                     
\begin{algorithmic} [1]  
  \SetKwProg{Fn}{Function}{}{}%                 % enter the algorithmic environment
  \SetKwInOut{Input}{Input}%
  \SetKwInOut{Output}{Output}%
%
  \Input{
    Zwei einfach verkettete Listen $h$ und $v$. \newline%
    $h$ enth{\"a}lt Endpunkte von horizontalen Segmenten. \newline%
    $v$ enth{\"a}lt vertikale Segmente. \newline%
    $h$ und $v$ sind aufsteigend gem{\"a}{\ss} $y$ bzw. $y_{unten}$ sortiert.%
  }%
  \Output{
    Berichtet alle Schnittpunkte zwischen Elementen aus $v$ und Elementen aus $h$.
  }%
  \Fn{IntersectAndReport($h$, $v$)}{
    \tcp{Keine Schnittpunkte m{\"o}glich}%
    \lIf{$h$ ist leer \textbf{oder} $v$ ist leer}{\Return}%
    \BlankLine%
    $currHor \leftarrow h$.head\;
    $currVert \leftarrow v$.head\;
    \While{$currHor \neq null$ und $currVert \neq null$}{
      \lIf{$currHor.y > currVert.y_{oben}$}{$currVert = currVert.next$}%
      \BlankLine%
      \eIf{$currHor.y \geq currVert.y_{unten}$}{
		    print($currVert.x$, $currHor.y$)\;
		    \BlankLine%
        $tmpHor \leftarrow currHor.next$\;
        \While{$tmpHor$ $\neq$ $null$ \textbf{und} $tmpHor.y < currVert.y_{oben}$}
        {
			    print($currVert.x$, $tmpHor.y$)\;
			    $tmpHor \leftarrow tmpHor.next$\;
		    }
		    $currVert \leftarrow currVer.next$\;
      }{
        \lIf{$currVert \neq null$}{$currHor \leftarrow currHor.next$}
      }
	  }
  }
\end{algorithmic}
\end{algorithm}

\end{document}
