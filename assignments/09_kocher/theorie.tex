\documentclass{article}

\usepackage{amsmath,amsthm,amssymb}
\usepackage{commath}
\usepackage{mathtools}
\usepackage{enumerate}
\usepackage{subcaption}
\usepackage{float}
\usepackage{tikz}
\usepackage[margin=1in]{geometry}
\usepackage{multicol}

\usetikzlibrary{positioning, shapes, automata, arrows}

\setlength{\parindent}{0pt}
\setlength{\parskip}{8pt}

\usepackage[utf8]{inputenc}
\begin{document}
\title{Assignment 9 \\ Advanced Algorithms \& Data Structures PS}%
\author{Christian Müller 1123410 \\ Daniel Kocher, 0926293}%
\maketitle

{\bfseries Aufgabe 18}%

Sei $Q$ eine Binomial Queue, die anfangs genau einen Binomialbaum $B_1$ mit den
Schl{\"u}sseln $13$ und $21$ enth{\"a}lt. F{\"u}gen Sie die Schl{\"u}ssel $3$,
$7$, $15$, $18$, $8$, $14$ und $27$ in die Queue ein. L{\"o}schen Sie
anschlie{\ss}end die Elemente $15$ und $27$ und wenden Sie
\texttt{decreasekey(18, 4)} an. Geben Sie nach jedem Schritt die resultierende
Queue an.

\tikzstyle{elem}=[draw, circle, thick, fill=blue!20, minimum size=10mm]
\tikzstyle{pointer}=[->, >=stealth, thick]
\tikzstyle{pointer-sibling}=[pointer, dotted]
\tikzstyle{pointer-child}=[pointer, solid]
\tikzstyle{pointer-parent}=[pointer, dashed]

Im Folgenden werden die einzelnen Schritte dargestellt, wobei f{\"u}r die
Child-Sibling-Parent Darstellung die folgenden Pfeile verwendet werden:
\begin{figure}[H]
  \centering
  \begin{tikzpicture}
    \draw[pointer-child] (0, 0) -- ++(0.5, 0) node[right] {\ldots Child-Pointer};
    \draw[pointer-parent] (5, 0) -- ++(0.5, 0) node[right] {\ldots Parent-Pointer};
    \draw[pointer-sibling] (10, 0) -- ++(0.5, 0) node[right] {\ldots Sibling-Pointer};
  \end{tikzpicture}
\end{figure}

\begin{multicols}{2}
\begin{figure}[H]
  \centering
  \begin{tikzpicture}
    \node[elem] (e-13) {$13$};
    \node[elem, below = 1 of e-13] (e-21) {$21$}
      edge [pointer-parent, bend right] (e-13);
    \draw[pointer-child] (e-13.south) -- (e-21.north);
  \end{tikzpicture}
  \caption{Ausgangs-Queue $Q$}
\end{figure}
\columnbreak
\begin{figure}[H]
  \centering
  \begin{tikzpicture}
    \node[elem] (e-13) {$13$};
    \node[elem, below = 1 of e-13] (e-21) {$21$}
      edge [pointer-parent, bend right] (e-13);
    \draw[pointer-child] (e-13.south) -- (e-21.north);

    \node[elem, right = 1 of e-13] (e-3) {$3$};

    \node[below right = 1 of e-3] (meld) {$\Longrightarrow$};
    \node[above = 0 of meld] (meld-t) {$meld$};

    \node[elem, right = 2 of e-3] (e-3) {$3$};
    \node[elem, right = 1 of e-3] (e-13) {$13$};
    \node[elem, below = 1 of e-13] (e-21) {$21$}
      edge [pointer-parent, bend right] (e-13);
    \draw[pointer-child] (e-13.south) -- (e-21.north);
    \draw[pointer-sibling] (e-3.east) -- (e-13.west);
  \end{tikzpicture}
  \caption{Einf{\"u}gen von $3$ ($meld$: $B_0$ vor $B_1$)} 
\end{figure}
\end{multicols}

\begin{figure}[H]
  \centering
  \begin{tikzpicture}
    \node[elem] (e-3) {$3$};
    \node[elem, right = 1 of e-3] (e-13) {$13$};
    \node[elem, below = 1 of e-13] (e-21) {$21$}
      edge[pointer-parent, bend right] (e-13);
    \draw[pointer-child] (e-13.south) -- (e-21.north);
    \draw[pointer-sibling] (e-3.east) -- (e-13.west);
    
    \node[elem, right = 1 of e-13] (e-7) {$7$};
    
    \node[below right = 1 of e-7] (meld-1) {$\Longrightarrow$};
    \node[above = 0 of meld-1] (meld-1-t) {$meld$};

    \node[elem, right = 2 of e-7] (e-3) {$3$};
    \node[elem, below = 1 of e-3] (e-7) {$7$}
      edge[pointer-parent, bend right] (e-3);
    \draw[pointer-child] (e-3.south) -- (e-7.north);
    \node[elem, right = 1 of e-3] (e-13) {$13$};
    \node[elem, below = 1 of e-13] (e-21) {$21$}
      edge[pointer-parent, bend right] (e-13);
    \draw[pointer-child] (e-13.south) -- (e-21.north);
    \draw[pointer-sibling] (e-3.east) -- (e-13.west);

    \node[below right = 1 of e-13] (meld-2) {$\Longrightarrow$};
    %\node[above = 0 of meld-2] (meld-2-t) {$meld$};

    \node[elem, right = 4 of e-13] (e-3) {$3$};
    \node[elem, below = 1 of e-3] (e-7) {$7$}
      edge[pointer-parent, bend right] (e-3);
    \node[elem, left = 1 of e-7] (e-13) {$13$}
      edge[pointer-parent, bend left] (e-3);
    \draw[pointer-sibling] (e-13.east) -- (e-7.west);
    \node[elem, below = 1 of e-13] (e-21) {$21$}
      edge[pointer-parent, bend right] (e-13);
    \draw[pointer-child] (e-13.south) -- (e-21.north);
    \draw[pointer-child] (e-3.south west) -- (e-13.north east);
  \end{tikzpicture}
  \caption{Einf{\"u}gen von $7$ ($meld$ vereinigt zweimal: $B_0$ und $B_1$)} 
\end{figure}

\begin{figure}[H]
  \centering
  \begin{tikzpicture}
    \node[elem] (e-3) {$3$};
    \node[elem, below = 1 of e-3] (e-7) {$7$}
      edge[pointer-parent, bend right] (e-3);
    \node[elem, left = 1 of e-7] (e-13) {$13$}
      edge[pointer-parent, bend left] (e-3);
    \draw[pointer-sibling] (e-13.east) -- (e-7.west);
    \node[elem, below = 1 of e-13] (e-21) {$21$}
      edge[pointer-parent, bend right] (e-13);
    \draw[pointer-child] (e-13.south) -- (e-21.north);
    \draw[pointer-child] (e-3.south west) -- (e-13.north east);

    \node[elem, right = 1 of e-3] (e-15) {$15$};

    \node[right = 3 of e-7] (meld) {$\Longrightarrow$};
    \node[above = 0 of meld] (meld-t) {$meld$};
    
    \node[elem, right = 5 of e-3] (e-15) {$15$};
    \node[elem, right = 3 of e-15] (e-3) {$3$};
    \draw[pointer-sibling] (e-15.east) -- (e-3.west);
    \node[elem, below = 1 of e-3] (e-7) {$7$}
      edge[pointer-parent, bend right] (e-3);
    \node[elem, left = 1 of e-7] (e-13) {$13$}
      edge[pointer-parent, bend left] (e-3);
    \draw[pointer-sibling] (e-13.east) -- (e-7.west);
    \node[elem, below = 1 of e-13] (e-21) {$21$}
      edge[pointer-parent, bend right] (e-13);
    \draw[pointer-child] (e-13.south) -- (e-21.north);
    \draw[pointer-child] (e-3.south west) -- (e-13.north east);
  \end{tikzpicture}
  \caption{Einf{\"u}gen von $15$ ($meld$: $B_0$ vor $B_2$)} 
\end{figure}

\begin{figure}[H]
  \centering
  \begin{tikzpicture}
    \node[elem] (e-15) {$15$};
    \node[elem, right = 3 of e-15] (e-3) {$3$};
    \draw[pointer-sibling] (e-15.east) -- (e-3.west);
    \node[elem, below = 1 of e-3] (e-7) {$7$}
      edge[pointer-parent, bend right] (e-3);
    \node[elem, left = 1 of e-7] (e-13) {$13$}
      edge[pointer-parent, bend left] (e-3);
    \draw[pointer-sibling] (e-13.east) -- (e-7.west);
    \node[elem, below = 1 of e-13] (e-21) {$21$}
      edge[pointer-parent, bend right] (e-13);
    \draw[pointer-child] (e-13.south) -- (e-21.north);
    \draw[pointer-child] (e-3.south west) -- (e-13.north east);

    \node[elem, right = 1 of e-3] (e-18) {$18$};

    \node[right = 3 of e-7] (meld) {$\Longrightarrow$};
    \node[above = 0 of meld] (meld-t) {$meld$};

    \node[elem, right = 3 of e-18] (e-15) {$15$};
    \node[elem, below = 1 of e-15] (e-18) {$18$}
      edge[pointer-parent, bend right] (e-15);
    \draw[pointer-child] (e-15.south) -- (e-18.north);
    \node[elem, right = 3 of e-15] (e-3) {$3$};
    \draw[pointer-sibling] (e-15.east) -- (e-3.west);
    \node[elem, below = 1 of e-3] (e-7) {$7$}
      edge[pointer-parent, bend right] (e-3);
    \node[elem, left = 1 of e-7] (e-13) {$13$}
      edge[pointer-parent, bend left] (e-3);
    \draw[pointer-sibling] (e-13.east) -- (e-7.west);
    \node[elem, below = 1 of e-13] (e-21) {$21$}
      edge[pointer-parent, bend right] (e-13);
    \draw[pointer-child] (e-13.south) -- (e-21.north);
    \draw[pointer-child] (e-3.south west) -- (e-13.north east);

  \end{tikzpicture}
  \caption{Einf{\"u}gen von $18$ ($meld$ vereinigt einmal: $B_0$)} 
\end{figure}

\begin{figure}[H]
  \centering
  \begin{tikzpicture}
    \node[elem] (e-15) {$15$};
    \node[elem, below = 1 of e-15] (e-18) {$18$}
      edge[pointer-parent, bend right] (e-15);
    \draw[pointer-child] (e-15.south) -- (e-18.north);
    \node[elem, right = 3 of e-15] (e-3) {$3$};
    \draw[pointer-sibling] (e-15.east) -- (e-3.west);
    \node[elem, below = 1 of e-3] (e-7) {$7$}
      edge[pointer-parent, bend right] (e-3);
    \node[elem, left = 1 of e-7] (e-13) {$13$}
      edge[pointer-parent, bend left] (e-3);
    \draw[pointer-sibling] (e-13.east) -- (e-7.west);
    \node[elem, below = 1 of e-13] (e-21) {$21$}
      edge[pointer-parent, bend right] (e-13);
    \draw[pointer-child] (e-13.south) -- (e-21.north);
    \draw[pointer-child] (e-3.south west) -- (e-13.north east);

    \node[elem, right = 1 of e-3] (e-8) {$8$};

    \node[right = 3 of e-7] (meld) {$\Longrightarrow$};
    \node[above = 0 of meld] (meld-t) {$meld$};

    \node[elem, right = 3 of e-8] (e-8) {$8$};
    \node[elem, right = 1 of e-8] (e-15) {$15$};
    \draw[pointer-sibling] (e-8.east) -- (e-15.west);
    \node[elem, below = 1 of e-15] (e-18) {$18$}
      edge[pointer-parent, bend right] (e-15);
    \draw[pointer-child] (e-15.south) -- (e-18.north);
    \node[elem, right = 3 of e-15] (e-3) {$3$};
    \draw[pointer-sibling] (e-15.east) -- (e-3.west);
    \node[elem, below = 1 of e-3] (e-7) {$7$}
      edge[pointer-parent, bend right] (e-3);
    \node[elem, left = 1 of e-7] (e-13) {$13$}
      edge[pointer-parent, bend left] (e-3);
    \draw[pointer-sibling] (e-13.east) -- (e-7.west);
    \node[elem, below = 1 of e-13] (e-21) {$21$}
      edge[pointer-parent, bend right] (e-13);
    \draw[pointer-child] (e-13.south) -- (e-21.north);
    \draw[pointer-child] (e-3.south west) -- (e-13.north east);
  \end{tikzpicture}
  \caption{Einf{\"u}gen von $8$ ($meld$: $B_0$ vor $B_1$ vor $B_2$)} 
\end{figure}

\begin{figure}[H]
  \centering
  \begin{tikzpicture}[scale = 0.8, every node/.style={scale=0.8}]
    \node[elem] (e-8) {$8$};
    \node[elem, right = 1 of e-8] (e-15) {$15$};
    \draw[pointer-sibling] (e-8.east) -- (e-15.west);
    \node[elem, below = 1 of e-15] (e-18) {$18$}
      edge[pointer-parent, bend right] (e-15);
    \draw[pointer-child] (e-15.south) -- (e-18.north);
    \node[elem, right = 3 of e-15] (e-3) {$3$};
    \draw[pointer-sibling] (e-15.east) -- (e-3.west);
    \node[elem, below = 1 of e-3] (e-7) {$7$}
      edge[pointer-parent, bend right] (e-3);
    \node[elem, left = 1 of e-7] (e-13) {$13$}
      edge[pointer-parent, bend left] (e-3);
    \draw[pointer-sibling] (e-13.east) -- (e-7.west);
    \node[elem, below = 1 of e-13] (e-21) {$21$}
      edge[pointer-parent, bend right] (e-13);
    \draw[pointer-child] (e-13.south) -- (e-21.north);
    \draw[pointer-child] (e-3.south west) -- (e-13.north east);

    \node[elem, right = 1 of e-3] (e-14) {$14$};

    \node[right = 3 of e-7] (meld-1) {$\Longrightarrow$};
    \node[above = 0 of meld-1] (meld-1-t) {$meld$};

    \node[elem, right = 3 of e-14] (e-8) {$8$};
    \node[elem, below = 1 of e-8] (e-14) {$14$}
      edge[pointer-parent, bend right] (e-8);
    \draw[pointer-child] (e-8.south) -- (e-14.north);
    \node[elem, right = 1 of e-8] (e-15) {$15$};
    \draw[pointer-sibling] (e-8.east) -- (e-15.west);
    \node[elem, below = 1 of e-15] (e-18) {$18$}
      edge[pointer-parent, bend right] (e-15);
    \draw[pointer-child] (e-15.south) -- (e-18.north);
    \node[elem, right = 3 of e-15] (e-3) {$3$};
    \draw[pointer-sibling] (e-15.east) -- (e-3.west);
    \node[elem, below = 1 of e-3] (e-7) {$7$}
      edge[pointer-parent, bend right] (e-3);
    \node[elem, left = 1 of e-7] (e-13) {$13$}
      edge[pointer-parent, bend left] (e-3);
    \draw[pointer-sibling] (e-13.east) -- (e-7.west);
    \node[elem, below = 1 of e-13] (e-21) {$21$}
      edge[pointer-parent, bend right] (e-13);
    \draw[pointer-child] (e-13.south) -- (e-21.north);
    \draw[pointer-child] (e-3.south west) -- (e-13.north east);

    \node[below = 0.5 of e-21] (meld-2) {$\Downarrow$};

    \node[elem, below = 4.5 of e-15] (e-8) {$8$};
    \node[elem, below = 1 of e-8] (e-14) {$14$}
      edge[pointer-parent, bend right] (e-8);
    \node[elem, left = 1 of e-14] (e-15) {$15$}
      edge[pointer-parent, bend left] (e-8);
    \draw[pointer-child] (e-8.south west) -- (e-15.north east);
    \draw[pointer-sibling] (e-15.east) -- (e-14.west);
    \node[elem, below = 1 of e-15] (e-18) {$18$}
      edge[pointer-parent, bend right] (e-15);
    \draw[pointer-child] (e-15.south) -- (e-18.north);
    \node[elem, right = 3 of e-8] (e-3) {$3$};
    \draw[pointer-sibling] (e-8.east) -- (e-3.west);
    \node[elem, below = 1 of e-3] (e-7) {$7$}
      edge[pointer-parent, bend right] (e-3);
    \node[elem, left = 1 of e-7] (e-13) {$13$}
      edge[pointer-parent, bend left] (e-3);
    \draw[pointer-sibling] (e-13.east) -- (e-7.west);
    \node[elem, below = 1 of e-13] (e-21) {$21$}
      edge[pointer-parent, bend right] (e-13);
    \draw[pointer-child] (e-13.south) -- (e-21.north);
    \draw[pointer-child] (e-3.south west) -- (e-13.north east);

    \node[left = 1 of e-15] (meld-3) {$\Longleftarrow$};

    \node[elem, left = 5 of e-8] (e-3) {$3$};
    \node[elem, below = 1 of e-3] (e-7) {$7$}
      edge[pointer-parent, bend right] (e-3);
    \node[elem, left = 1 of e-7] (e-13) {$13$}
      edge[pointer-parent, bend right] (e-3);
    \node[elem, left = 1 of e-13] (e-8) {$8$}
      edge[pointer-parent, bend left] (e-3);
    \draw[pointer-sibling] (e-13.east) -- (e-7.west);
    \node[elem, below = 1 of e-13] (e-21) {$21$}
      edge[pointer-parent, bend right] (e-13);
    \draw[pointer-child] (e-13.south) -- (e-21.north);
    \draw[pointer-child] (e-3.south west) -- (e-8.north east);
    \draw[pointer-sibling] (e-8.east) -- (e-13.west);
    \node[elem, below = 1 of e-8] (e-14) {$14$}
      edge[pointer-parent, bend right] (e-8);
    \node[elem, left = 1 of e-14] (e-15) {$15$}
      edge[pointer-parent, bend right] (e-8);
    \draw[pointer-sibling] (e-15.east) -- (e-14.west);
    \node[elem, below = 1 of e-15] (e-18) {$18$}
      edge[pointer-parent, bend right] (e-15);
    \draw[pointer-child] (e-8.south west) -- (e-15.north east);
    \draw[pointer-child] (e-15.south) -- (e-18.north);
  \end{tikzpicture}
  \caption{
    Einf{\"u}gen von $14$ ($meld$ vereinigt dreimal: $B_0$, $B_1$ and $B_2$)
  } 
\end{figure}

\begin{figure}[H]
  \centering
  \begin{tikzpicture}[scale = 0.8, every node/.style={scale=0.8}]
    \node[elem] (e-3) {$3$};
    \node[elem, below = 1 of e-3] (e-7) {$7$}
      edge[pointer-parent, bend right] (e-3);
    \node[elem, left = 1 of e-7] (e-13) {$13$}
      edge[pointer-parent, bend right] (e-3);
    \node[elem, left = 1 of e-13] (e-8) {$8$}
      edge[pointer-parent, bend left] (e-3);
    \draw[pointer-sibling] (e-13.east) -- (e-7.west);
    \node[elem, below = 1 of e-13] (e-21) {$21$}
      edge[pointer-parent, bend right] (e-13);
    \draw[pointer-child] (e-13.south) -- (e-21.north);
    \draw[pointer-child] (e-3.south west) -- (e-8.north east);
    \draw[pointer-sibling] (e-8.east) -- (e-13.west);
    \node[elem, below = 1 of e-8] (e-14) {$14$}
      edge[pointer-parent, bend right] (e-8);
    \node[elem, left = 1 of e-14] (e-15) {$15$}
      edge[pointer-parent, bend right] (e-8);
    \draw[pointer-sibling] (e-15.east) -- (e-14.west);
    \node[elem, below = 1 of e-15] (e-18) {$18$}
      edge[pointer-parent, bend right] (e-15);
    \draw[pointer-child] (e-8.south west) -- (e-15.north east);
    \draw[pointer-child] (e-15.south) -- (e-18.north);

    \node[elem, right = 1 of e-3] (e-27) {$27$};

    \node[right = 3 of e-7] (meld) {$\Longrightarrow$};
    \node[above = 0 of meld] (meld-t) {$meld$};

    \node[elem, right = 5 of e-3] (e-27) {$27$};  
    \node[elem, right = 5 of e-27] (e-3) {$3$};
    \draw[pointer-sibling] (e-27.east) -- (e-3.west);
    \node[elem, below = 1 of e-3] (e-7) {$7$}
      edge[pointer-parent, bend right] (e-3);
    \node[elem, left = 1 of e-7] (e-13) {$13$}
      edge[pointer-parent, bend right] (e-3);
    \node[elem, left = 1 of e-13] (e-8) {$8$}
      edge[pointer-parent, bend left] (e-3);
    \draw[pointer-sibling] (e-13.east) -- (e-7.west);
    \node[elem, below = 1 of e-13] (e-21) {$21$}
      edge[pointer-parent, bend right] (e-13);
    \draw[pointer-child] (e-13.south) -- (e-21.north);
    \draw[pointer-child] (e-3.south west) -- (e-8.north east);
    \draw[pointer-sibling] (e-8.east) -- (e-13.west);
    \node[elem, below = 1 of e-8] (e-14) {$14$}
      edge[pointer-parent, bend right] (e-8);
    \node[elem, left = 1 of e-14] (e-15) {$15$}
      edge[pointer-parent, bend right] (e-8);
    \draw[pointer-sibling] (e-15.east) -- (e-14.west);
    \node[elem, below = 1 of e-15] (e-18) {$18$}
      edge[pointer-parent, bend right] (e-15);
    \draw[pointer-child] (e-8.south west) -- (e-15.north east);
    \draw[pointer-child] (e-15.south) -- (e-18.north);  
  \end{tikzpicture}
  \caption{
    Einf{\"u}gen von $14$ ($meld$ vereinigt dreimal: $B_0$, $B_1$ and $B_2$)
  } 
\end{figure}

\begin{figure}[H]
  \centering
  \begin{tikzpicture}[scale = 0.8, every node/.style={scale=0.8}]
    \node[elem] (e-27) {$27$};  
    \node[elem, right = 5 of e-27] (e-3) {$3$};
    \draw[pointer-sibling] (e-27.east) -- (e-3.west);
    \node[elem, below = 1 of e-3] (e-7) {$7$}
      edge[pointer-parent, bend right] (e-3);
    \node[elem, left = 1 of e-7] (e-13) {$13$}
      edge[pointer-parent, bend right] (e-3);
    \node[elem, left = 1 of e-13] (e-8) {$8$}
      edge[pointer-parent, bend left] (e-3);
    \draw[pointer-sibling] (e-13.east) -- (e-7.west);
    \node[elem, below = 1 of e-13] (e-21) {$21$}
      edge[pointer-parent, bend right] (e-13);
    \draw[pointer-child] (e-13.south) -- (e-21.north);
    \draw[pointer-child] (e-3.south west) -- (e-8.north east);
    \draw[pointer-sibling] (e-8.east) -- (e-13.west);
    \node[elem, below = 1 of e-8] (e-14) {$14$}
      edge[pointer-parent, bend right] (e-8);
    \node[elem, left = 1 of e-14] (e-15) {$-\infty$}
      edge[pointer-parent, bend right] (e-8);
    \draw[pointer-sibling] (e-15.east) -- (e-14.west);
    \node[elem, below = 1 of e-15] (e-18) {$18$}
      edge[pointer-parent, bend right] (e-15);
    \draw[pointer-child] (e-8.south west) -- (e-15.north east);
    \draw[pointer-child] (e-15.south) -- (e-18.north);

    \node[right = 1 of e-7] {$\Longrightarrow$};

    \node[elem, right = 3 of e-3] (e-27) {$27$};  
    \node[elem, right = 5 of e-27] (e-3) {$3$};
    \draw[pointer-sibling] (e-27.east) -- (e-3.west);
    \node[elem, below = 1 of e-3] (e-7) {$7$}
      edge[pointer-parent, bend right] (e-3);
    \node[elem, left = 1 of e-7] (e-13) {$13$}
      edge[pointer-parent, bend right] (e-3);
    \node[elem, left = 1 of e-13] (e-8) {$-\infty$}
      edge[pointer-parent, bend left] (e-3);
    \draw[pointer-sibling] (e-13.east) -- (e-7.west);
    \node[elem, below = 1 of e-13] (e-21) {$21$}
      edge[pointer-parent, bend right] (e-13);
    \draw[pointer-child] (e-13.south) -- (e-21.north);
    \draw[pointer-child] (e-3.south west) -- (e-8.north east);
    \draw[pointer-sibling] (e-8.east) -- (e-13.west);
    \node[elem, below = 1 of e-8] (e-14) {$14$}
      edge[pointer-parent, bend right] (e-8);
    \node[elem, left = 1 of e-14] (e-15) {$8$}
      edge[pointer-parent, bend right] (e-8);
    \draw[pointer-sibling] (e-15.east) -- (e-14.west);
    \node[elem, below = 1 of e-15] (e-18) {$18$}
      edge[pointer-parent, bend right] (e-15);
    \draw[pointer-child] (e-8.south west) -- (e-15.north east);
    \draw[pointer-child] (e-15.south) -- (e-18.north);

    \node[below = 3 of e-21] {$\Downarrow$};

    \node[elem, below = 8 of e-27] (e-27) {$27$};  
    \node[elem, right = 5 of e-27] (e-3) {$-\infty$};
    \draw[pointer-sibling] (e-27.east) -- (e-3.west);
    \node[elem, below = 1 of e-3] (e-7) {$7$}
      edge[pointer-parent, bend right] (e-3);
    \node[elem, left = 1 of e-7] (e-13) {$13$}
      edge[pointer-parent, bend right] (e-3);
    \node[elem, left = 1 of e-13] (e-8) {$3$}
      edge[pointer-parent, bend left] (e-3);
    \draw[pointer-sibling] (e-13.east) -- (e-7.west);
    \node[elem, below = 1 of e-13] (e-21) {$21$}
      edge[pointer-parent, bend right] (e-13);
    \draw[pointer-child] (e-13.south) -- (e-21.north);
    \draw[pointer-child] (e-3.south west) -- (e-8.north east);
    \draw[pointer-sibling] (e-8.east) -- (e-13.west);
    \node[elem, below = 1 of e-8] (e-14) {$14$}
      edge[pointer-parent, bend right] (e-8);
    \node[elem, left = 1 of e-14] (e-15) {$8$}
      edge[pointer-parent, bend right] (e-8);
    \draw[pointer-sibling] (e-15.east) -- (e-14.west);
    \node[elem, below = 1 of e-15] (e-18) {$18$}
      edge[pointer-parent, bend right] (e-15);
    \draw[pointer-child] (e-8.south west) -- (e-15.north east);
    \draw[pointer-child] (e-15.south) -- (e-18.north);
  \end{tikzpicture}
  \caption{
    L{\"o}schen von $15$: Ersetze $15$ mit $-\infty$ und lasse diesen Knoten nach
    oben wandern.
  } 
\end{figure}

\clearpage

\begin{figure}[H]
  \centering
  \begin{tikzpicture}[scale = 0.8, every node/.style={scale=0.8}]
    \node[elem] (e-27) {$27$};  
    \node[elem, right = 5 of e-27] (e-3) {$-\infty$};
    \draw[pointer-sibling] (e-27.east) -- (e-3.west);
    \node[elem, below = 1 of e-3] (e-7) {$7$}
      edge[pointer-parent, bend right] (e-3);
    \node[elem, left = 1 of e-7] (e-13) {$13$}
      edge[pointer-parent, bend right] (e-3);
    \node[elem, left = 1 of e-13] (e-8) {$3$}
      edge[pointer-parent, bend left] (e-3);
    \draw[pointer-sibling] (e-13.east) -- (e-7.west);
    \node[elem, below = 1 of e-13] (e-21) {$21$}
      edge[pointer-parent, bend right] (e-13);
    \draw[pointer-child] (e-13.south) -- (e-21.north);
    \draw[pointer-child] (e-3.south west) -- (e-8.north east);
    \draw[pointer-sibling] (e-8.east) -- (e-13.west);
    \node[elem, below = 1 of e-8] (e-14) {$14$}
      edge[pointer-parent, bend right] (e-8);
    \node[elem, left = 1 of e-14] (e-15) {$8$}
      edge[pointer-parent, bend right] (e-8);
    \draw[pointer-sibling] (e-15.east) -- (e-14.west);
    \node[elem, below = 1 of e-15] (e-18) {$18$}
      edge[pointer-parent, bend right] (e-15);
    \draw[pointer-child] (e-8.south west) -- (e-15.north east);
    \draw[pointer-child] (e-15.south) -- (e-18.north);

    \node[right = 1 of e-7] (deletemin) {$\Longrightarrow$};
    \node[above = 0 of deletemin] (deletemin-t) {$deletemin$};

    \node[elem, right = 3 of e-3] (e-27) {$27$};

    \node[elem, right = 1 of e-27] (e-7) {$7$};
    \node[elem, right = 1 of e-7] (e-13) {$13$};
    \node[elem, below = 1 of e-13] (e-21) {$21$}
      edge[pointer-parent, bend right] (e-13);
    \draw[pointer-child] (e-13.south) -- (e-21.north);
    \draw[pointer-sibling] (e-7.east) -- (e-13.west);
    \node[elem, right = 3 of e-13] (e-3) {$3$};
    \node[elem, below = 1 of e-3] (e-14) {$14$}
      edge[pointer-parent, bend right] (e-3);
    \node[elem, left = 1 of e-14] (e-8) {$8$}
      edge[pointer-parent, bend right] (e-3);
    \node[elem, below = 1 of e-8] (e-18) {$18$}
      edge[pointer-parent, bend right] (e-8);
    \draw[pointer-child] (e-8.south) -- (e-18.north);
    \draw[pointer-child] (e-3.south west) -- (e-8.north east);
    \draw[pointer-sibling] (e-13.east) -- (e-3.west);

    \node[below = 1.5 of e-18] (meld-1) {$\Downarrow$};
    \node[right = 0 of meld-1] (meld-1-t) {$meld$};

    \node[elem, below = 6.5 of e-7] (e-7) {$7$};
    \node[elem, below = 1 of e-7] (e-27) {$27$}
      edge[pointer-parent, bend right] (e-7);
    \draw[pointer-child] (e-7.south) -- (e-27.north);
    \node[elem, right = 1 of e-7] (e-13) {$13$};
    \node[elem, below = 1 of e-13] (e-21) {$21$}
      edge[pointer-parent, bend right] (e-13);
    \draw[pointer-child] (e-13.south) -- (e-21.north);
    \draw[pointer-sibling] (e-7.east) -- (e-13.west);
    \node[elem, right = 3 of e-13] (e-3) {$3$};
    \node[elem, below = 1 of e-3] (e-14) {$14$}
      edge[pointer-parent, bend right] (e-3);
    \node[elem, left = 1 of e-14] (e-8) {$8$}
      edge[pointer-parent, bend right] (e-3);
    \draw[pointer-child] (e-3.south west) -- (e-8.north east);
    \node[elem, below = 1 of e-8] (e-18) {$18$}
      edge[pointer-parent, bend right] (e-8);
    \draw[pointer-child] (e-8.south) -- (e-18.north);
    \draw[pointer-sibling] (e-8.east) -- (e-14.west);
    \draw[pointer-sibling] (e-13.east) -- (e-3.west);
  
    \node[left = 1 of e-27] (meld-2) {$\Longleftarrow$};

    \node[elem, left = 8 of e-7] (e-7) {$7$};
    \node[elem, below = 1 of e-7] (e-27) {$27$}
      edge[pointer-parent, bend right] (e-7);
    \node[elem, left = 1 of e-27] (e-13) {$13$}
      edge[pointer-parent, bend right] (e-7);
    \draw[pointer-child] (e-7.south west) -- (e-13.north east);
    \node[elem, below = 1 of e-13] (e-21) {$21$}
      edge[pointer-parent, bend right] (e-13);
    \draw[pointer-child] (e-13.south) -- (e-21.north);
    \draw[pointer-sibling] (e-13.east) -- (e-27.west);
    \node[elem, right = 3 of e-7] (e-3) {$3$};
    \node[elem, below = 1 of e-3] (e-14) {$14$}
      edge[pointer-parent, bend right] (e-3);
    \node[elem, left = 1 of e-14] (e-8) {$8$}
      edge[pointer-parent, bend right] (e-3);
    \draw[pointer-child] (e-3.south west) -- (e-8.north east);
    \node[elem, below = 1 of e-8] (e-18) {$18$}
      edge[pointer-parent, bend right] (e-8);
    \draw[pointer-child] (e-8.south) -- (e-18.north);
    \draw[pointer-sibling] (e-8.east) -- (e-14.west);
    \draw[pointer-sibling] (e-7.east) -- (e-3.west);

    \node[below = 1 of e-18] (meld-3) {$\Downarrow$};

    \node[elem, below = 6 of e-3] (e-3) {$3$};
    \node[elem, below = 1 of e-3] (e-14) {$14$}
      edge[pointer-parent, bend right] (e-3);
    \node[elem, left = 1 of e-14] (e-8) {$8$}
      edge[pointer-parent, bend right] (e-3);
    \draw[pointer-sibling] (e-8.east) -- (e-14.west);
    \node[elem, left = 1 of e-8] (e-7) {$7$}
      edge[pointer-parent, bend left] (e-3);
    \draw[pointer-sibling] (e-7.east) -- (e-8.west);
    \draw[pointer-child] (e-3.south west) -- (e-7.north east);
    \node[elem, below = 1 of e-8] (e-18) {$18$}
      edge[pointer-parent, bend right] (e-8);
    \draw[pointer-child] (e-8.south) -- (e-18.north);
    \node[elem, left = 1 of e-18] (e-27) {$27$}
      edge[pointer-parent, bend right] (e-7);
    \node[elem, left = 1 of e-27] (e-13) {$13$}
      edge[pointer-parent, bend right] (e-7);
    \draw[pointer-sibling] (e-13.east) -- (e-27.west);
    \draw[pointer-child] (e-7.south west) -- (e-13.north east);
    \node[elem, below = 1 of e-13] (e-21) {$21$}
      edge[pointer-parent, bend right] (e-13);
    \draw[pointer-child] (e-13.south) -- (e-21.north);
  \end{tikzpicture}
  \caption{
    \texttt{deletemin()} ($-\infty$ hat minimalen Schl{\"u}ssel in der
    Wurzelliste $\Rightarrow$ entferne $-\infty$ aus $Q$ (liefert $Q'$); Drehe
    Reihenfolge der S{\"o}hne von $-\infty$ um (liefert $Q''$); $Q'.meld(Q'')$)
  } 
\end{figure}

\begin{figure}[H]
  \centering
  \begin{tikzpicture}[scale = 0.8, every node/.style={scale=0.8}]
    \node[elem] (e-3) {$3$};
    \node[elem, below = 1 of e-3] (e-14) {$14$}
      edge[pointer-parent, bend right] (e-3);
    \node[elem, left = 1 of e-14] (e-8) {$8$}
      edge[pointer-parent, bend right] (e-3);
    \draw[pointer-sibling] (e-8.east) -- (e-14.west);
    \node[elem, left = 1 of e-8] (e-7) {$7$}
      edge[pointer-parent, bend left] (e-3);
    \draw[pointer-sibling] (e-7.east) -- (e-8.west);
    \draw[pointer-child] (e-3.south west) -- (e-7.north east);
    \node[elem, below = 1 of e-8] (e-18) {$18$}
      edge[pointer-parent, bend right] (e-8);
    \draw[pointer-child] (e-8.south) -- (e-18.north);
    \node[elem, left = 1 of e-18] (e-27) {$-\infty$}
      edge[pointer-parent, bend right] (e-7);
    \node[elem, left = 1 of e-27] (e-13) {$13$}
      edge[pointer-parent, bend right] (e-7);
    \draw[pointer-sibling] (e-13.east) -- (e-27.west);
    \draw[pointer-child] (e-7.south west) -- (e-13.north east);
    \node[elem, below = 1 of e-13] (e-21) {$21$}
      edge[pointer-parent, bend right] (e-13);
    \draw[pointer-child] (e-13.south) -- (e-21.north);

    \node[right = 1 of e-14] {$\Longrightarrow$};

    \node[elem, right = 8 of e-3] (e-3) {$3$};
    \node[elem, below = 1 of e-3] (e-14) {$14$}
      edge[pointer-parent, bend right] (e-3);
    \node[elem, left = 1 of e-14] (e-8) {$8$}
      edge[pointer-parent, bend right] (e-3);
    \draw[pointer-sibling] (e-8.east) -- (e-14.west);
    \node[elem, left = 1 of e-8] (e-7) {$-\infty$}
      edge[pointer-parent, bend left] (e-3);
    \draw[pointer-sibling] (e-7.east) -- (e-8.west);
    \draw[pointer-child] (e-3.south west) -- (e-7.north east);
    \node[elem, below = 1 of e-8] (e-18) {$18$}
      edge[pointer-parent, bend right] (e-8);
    \draw[pointer-child] (e-8.south) -- (e-18.north);
    \node[elem, left = 1 of e-18] (e-27) {$7$}
      edge[pointer-parent, bend right] (e-7);
    \node[elem, left = 1 of e-27] (e-13) {$13$}
      edge[pointer-parent, bend right] (e-7);
    \draw[pointer-sibling] (e-13.east) -- (e-27.west);
    \draw[pointer-child] (e-7.south west) -- (e-13.north east);
    \node[elem, below = 1 of e-13] (e-21) {$21$}
      edge[pointer-parent, bend right] (e-13);
    \draw[pointer-child] (e-13.south) -- (e-21.north);

    \node[below = 2.5 of e-18] {$\Downarrow$};

    \node[elem, below = 7 of e-3] (e-3) {$-\infty$};
    \node[elem, below = 1 of e-3] (e-14) {$14$}
      edge[pointer-parent, bend right] (e-3);
    \node[elem, left = 1 of e-14] (e-8) {$8$}
      edge[pointer-parent, bend right] (e-3);
    \draw[pointer-sibling] (e-8.east) -- (e-14.west);
    \node[elem, left = 1 of e-8] (e-7) {$3$}
      edge[pointer-parent, bend left] (e-3);
    \draw[pointer-sibling] (e-7.east) -- (e-8.west);
    \draw[pointer-child] (e-3.south west) -- (e-7.north east);
    \node[elem, below = 1 of e-8] (e-18) {$18$}
      edge[pointer-parent, bend right] (e-8);
    \draw[pointer-child] (e-8.south) -- (e-18.north);
    \node[elem, left = 1 of e-18] (e-27) {$7$}
      edge[pointer-parent, bend right] (e-7);
    \node[elem, left = 1 of e-27] (e-13) {$13$}
      edge[pointer-parent, bend right] (e-7);
    \draw[pointer-sibling] (e-13.east) -- (e-27.west);
    \draw[pointer-child] (e-7.south west) -- (e-13.north east);
    \node[elem, below = 1 of e-13] (e-21) {$21$}
      edge[pointer-parent, bend right] (e-13);
    \draw[pointer-child] (e-13.south) -- (e-21.north);
  \end{tikzpicture}
  \caption{
    L{\"o}schen von $27$: Ersetze $27$ mit $-\infty$ und lasse diesen Knoten nach
    oben wandern.
  } 
\end{figure}

\begin{figure}[H]
  \centering
  \begin{tikzpicture}[scale = 0.8, every node/.style={scale=0.8}]
    \node[elem, below = 7 of e-3] (e-3) {$-\infty$};
    \node[elem, below = 1 of e-3] (e-14) {$14$}
      edge[pointer-parent, bend right] (e-3);
    \node[elem, left = 1 of e-14] (e-8) {$8$}
      edge[pointer-parent, bend right] (e-3);
    \draw[pointer-sibling] (e-8.east) -- (e-14.west);
    \node[elem, left = 1 of e-8] (e-7) {$3$}
      edge[pointer-parent, bend left] (e-3);
    \draw[pointer-sibling] (e-7.east) -- (e-8.west);
    \draw[pointer-child] (e-3.south west) -- (e-7.north east);
    \node[elem, below = 1 of e-8] (e-18) {$18$}
      edge[pointer-parent, bend right] (e-8);
    \draw[pointer-child] (e-8.south) -- (e-18.north);
    \node[elem, left = 1 of e-18] (e-27) {$7$}
      edge[pointer-parent, bend right] (e-7);
    \node[elem, left = 1 of e-27] (e-13) {$13$}
      edge[pointer-parent, bend right] (e-7);
    \draw[pointer-sibling] (e-13.east) -- (e-27.west);
    \draw[pointer-child] (e-7.south west) -- (e-13.north east);
    \node[elem, below = 1 of e-13] (e-21) {$21$}
      edge[pointer-parent, bend right] (e-13);
    \draw[pointer-child] (e-13.south) -- (e-21.north);

    \node[right = 1 of e-14] (deletemin) {$\Longrightarrow$};
    \node[above = 0 of deletemin] (deletemin-t) {$deletemin$};

    \node[elem, right = 3 of e-3] (e-14) {$14$};
    \node[elem, right = 1 of e-14] (e-8) {$8$};
    \draw[pointer-sibling] (e-14.east) -- (e-8.west);
    \node[elem, below = 1 of e-8] (e-18) {$18$}
      edge[pointer-parent, bend right] (e-8);
    \draw[pointer-child] (e-8.south) -- (e-18.north);
    \node[elem, right = 3 of e-8] (e-3) {$3$};
    \draw[pointer-sibling] (e-8.east) -- (e-3.west);
    \node[elem, below = 1 of e-3] (e-7) {$7$}
      edge[pointer-parent, bend right] (e-3);
    \node[elem, left = 1 of e-7] (e-13) {$13$}
      edge[pointer-parent, bend right] (e-3);
    \draw[pointer-sibling] (e-13.east) -- (e-7.west);
    \draw[pointer-child] (e-3.south west) -- (e-13.north east);
    \node[elem, below = 1 of e-13] (e-21) {$21$}
      edge[pointer-parent, bend right] (e-13);
    \draw[pointer-child] (e-13.south) -- (e-21.north);
  \end{tikzpicture}
  \caption{
    \texttt{deletemin()} ($-\infty$ hat minimalen Schl{\"u}ssel in der
    Wurzelliste $\Rightarrow$ entferne $-\infty$ aus $Q$ (liefert $Q'$); Drehe
    Reihenfolge der S{\"o}hne von $-\infty$ um (liefert $Q''$); $Q'.meld(Q'')$).
    In diesem Fall ist $Q'$ leer, wodurch nurmehr $Q''$ bleibt.
  } 
\end{figure}

\begin{figure}[H]
  \centering
  \begin{tikzpicture}[scale = 0.8, every node/.style={scale=0.8}]
    \node[elem] (e-14) {$14$};
    \node[elem, right = 1 of e-14] (e-8) {$8$};
    \draw[pointer-sibling] (e-14.east) -- (e-8.west);
    \node[elem, below = 1 of e-8] (e-18) {$18$}
      edge[pointer-parent, bend right] (e-8);
    \draw[pointer-child] (e-8.south) -- (e-18.north);
    \node[elem, right = 3 of e-8] (e-3) {$3$};
    \draw[pointer-sibling] (e-8.east) -- (e-3.west);
    \node[elem, below = 1 of e-3] (e-7) {$7$}
      edge[pointer-parent, bend right] (e-3);
    \node[elem, left = 1 of e-7] (e-13) {$13$}
      edge[pointer-parent, bend right] (e-3);
    \draw[pointer-sibling] (e-13.east) -- (e-7.west);
    \draw[pointer-child] (e-3.south west) -- (e-13.north east);
    \node[elem, below = 1 of e-13] (e-21) {$21$}
      edge[pointer-parent, bend right] (e-13);
    \draw[pointer-child] (e-13.south) -- (e-21.north);

    \node[right = 1 of e-7] {$\Longrightarrow$};

    \node[elem, right = 3 of e-3] (e-14) {$14$};
    \node[elem, right = 1 of e-14] (e-8) {$8$};
    \draw[pointer-sibling] (e-14.east) -- (e-8.west);
    \node[elem, below = 1 of e-8] (e-4) {$4$}
      edge[pointer-parent, bend right] (e-8);
    \draw[pointer-child] (e-8.south) -- (e-4.north);
    \node[elem, right = 3 of e-8] (e-3) {$3$};
    \draw[pointer-sibling] (e-8.east) -- (e-3.west);
    \node[elem, below = 1 of e-3] (e-7) {$7$}
      edge[pointer-parent, bend right] (e-3);
    \node[elem, left = 1 of e-7] (e-13) {$13$}
      edge[pointer-parent, bend right] (e-3);
    \draw[pointer-sibling] (e-13.east) -- (e-7.west);
    \draw[pointer-child] (e-3.south west) -- (e-13.north east);
    \node[elem, below = 1 of e-13] (e-21) {$21$}
      edge[pointer-parent, bend right] (e-13);
    \draw[pointer-child] (e-13.south) -- (e-21.north);  

    \node[below = 1 of e-21]  {$\Downarrow$};

    \node[elem, below = 5.5 of e-14] (e-14) {$14$};
    \node[elem, right = 1 of e-14] (e-8) {$4$};
    \draw[pointer-sibling] (e-14.east) -- (e-8.west);
    \node[elem, below = 1 of e-8] (e-4) {$8$}
      edge[pointer-parent, bend right] (e-8);
    \draw[pointer-child] (e-8.south) -- (e-4.north);
    \node[elem, right = 3 of e-8] (e-3) {$3$};
    \draw[pointer-sibling] (e-8.east) -- (e-3.west);
    \node[elem, below = 1 of e-3] (e-7) {$7$}
      edge[pointer-parent, bend right] (e-3);
    \node[elem, left = 1 of e-7] (e-13) {$13$}
      edge[pointer-parent, bend right] (e-3);
    \draw[pointer-sibling] (e-13.east) -- (e-7.west);
    \draw[pointer-child] (e-3.south west) -- (e-13.north east);
    \node[elem, below = 1 of e-13] (e-21) {$21$}
      edge[pointer-parent, bend right] (e-13);
    \draw[pointer-child] (e-13.south) -- (e-21.north);  
  \end{tikzpicture}
  \caption{
    \texttt{decreasekey(18, 4)} (1. $v.entry.key = k$; 2. $v.entry$ nach oben
    steigen lassen in dem geg. Baum, bis die Heapbedingung erf{\"u}llt ist). In
    diesem Fall ersetzen wir $18$ durch $4$ und lassen $4$ um eine Ebene nach
    oben steigen.
  } 
\end{figure}

\end{document}