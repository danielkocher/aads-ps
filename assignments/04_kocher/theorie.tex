\documentclass{article}

\usepackage{amsmath,amsthm,amssymb}
\usepackage{commath}
\usepackage{mathtools}
\usepackage[linesnumbered]{algorithm2e}
\usepackage[margin=.75in]{geometry}

\setlength{\parindent}{0pt}
\setlength{\parskip}{8pt}

\usepackage[utf8]{inputenc}
\begin{document}
\title{Assignment 4 \\ Advanced Algorithms \& Data Structures PS}%
\author{Christian Müller 1123410 \\ Daniel Kocher, 0926293}%
\maketitle

{\noindent\bfseries Aufgabe 7}%
\medskip%

\begin{algorithm}
  \SetKwProg{Fn}{Function}{}{}%
  \SetKwInOut{Input}{Input}%
%
  \Input{
    Boolsche Matrizen $A$, $B \in \{ 0, 1 \}^{n \times n}$
  }%
  \Fn{BPWM($A$, $B$)}{%
    \tcp{berechnet eine Zeugenmatrix $W$ f{\"u}r $P = A * B$}%
    $W = -AB$ \tcp{$W_{ij} < 0 \Longleftrightarrow$ Zeuge f{\"u}r $P_{ij}$ muss noch gefunden werden}
    \tcp{teste alle Zweierpotenzen $r = 2^t$ zwischen $1$ und $n$}
    \For{$t = 0, \ldots, \left\lfloor log(n) \right\rfloor$}{
      $r = 2^t$\;
      \Repeat{$\left\lceil 6.95 log(n) \right\rceil$ times done}{ \tcp{f{\"u}r $O(log(n))$ Runden}
        w{\"a}hle $R \subseteq \{ 1, \ldots, n \}$ mit $|R| = r$\;
        berechne $A^R$ und $B^R$ \tcp{vgl. Lemma (Folie 21)}
        $Z = A^R B^R$\;
        \tcp{teste ob neuer Zeuge f{\"u}r $P_{ij}$ gefunden wurde}
        \ForAll{$(i, j)$}{
          \lIf{$W_{ij}$ und $Z_{ij}$ ist Zeuge f{\"u}r $P_{ij}$}{$W_{ij} = Z_{ij}$}
        }
      }
    }
    \ForAll{$(i, j)$}{
      \If{$W_{ij} < 0$}{
        \tcp{kein Zeuge f{\"u}r $P_{ij}$ gefunden}
        berechne $W_{ij}$ mit dem trivialen Algorithmus
      }
    }
  }%
\end{algorithm}

\noindent
Der $BPWM$-Algorithmus entspricht grunds{\"a}tzlich genau dem Algorithmus, der
in der Vorlesung vorgestellt wurde (Folie 23 vom 07.04.2016). Es gibt lediglich
eine {\"A}nderung: der konstante Faktor der inneren Schleife wurde von $3.77$ auf
$6.95$ erh{\"o}ht. Genau diese {\"A}nderung hat dann auch den gew{\"u}nschten
Effekt, dass der $BPWM$-Algorithmus die Laufzeit
$O \left( MM \left( n \right) \cdot log^2(n) \right)$ mit
Wahrscheinlichkeit $1 - n^{-2}$ garantiert. Dies wird nun bewiesen.

\clearpage

{\bfseries F{\"u}r den urspr{\"u}nglichen Fall:} \newline
Aus der Vorlesung wissen wir, dass die Wahrscheinlichkeit sehr gering ist, dass
der $BPWM$-Algorithmus keinen eindeutigen Zeugen findet:

\begin{equation}
  \left( 1 - \frac{1}{2e} \right) \quad \text{\ldots Wahrscheinlichkeit, dass
    kein eindeutiger Zeuge gefunden wird (ein Lauf)}
\end{equation}

\begin{equation}
  \left( 1 - \frac{1}{2e} \right)^{\left\lceil 3.77log(n) \right\rceil} \quad
    \text{\ldots Wahrscheinlichkeit, dass kein eindeutiger Zeuge gefunden wird
    ($\left\lceil 3.77log(n) \right\rceil$ L{\"a}ufe)}
\end{equation}

Ist $P_{ij} = 1$, so wird im randomisierten Teil des $BPWM$-Algorithmus ein Zeuge
mit Wahrscheinlichkeit $\geq 1 - \frac{1}{n}$ gefunden.
$\Longleftrightarrow$ da mit Wahrscheinlichkeit

\begin{equation}
  \left( 1 - \frac{1}{2e} \right)^{\left\lceil 3.77log(n) \right\rceil} \leq \frac{1}{n}
\end{equation}

kein eindeutiger Zeuge gefunden wird, wird also mit

\begin{equation}
  1 - \left( 1 - \frac{1}{2e} \right)^{\left\lceil 3.77log(n) \right\rceil} \geq 1 - \frac{1}{n} = 1 - n^{-1}
\end{equation}

ein eindeutiger Zeuge gefunden.

Diese Schranke soll nun durch Erh{\"o}hung des konstanten Faktor von $3.77$ auf
$6.95$ so angepasst werden, dass mit Wahrscheinlichkeit

\begin{equation}
  \geq 1 - \frac{1}{n^2} = 1 - n^{-2}
\end{equation}

die Laufzeit von $O \left( MM \left( n \right) \cdot log^2(n) \right)$ garantiert
werden kann.

{\bfseries F{\"u}r den angepassten Fall:} \newline
Die Laufzeit $O \left( MM \left( n \right) \cdot log^2(n) \right)$ folgt, wenn
der triviale Algorithmus nicht zu oft aufgerufen werden muss.

Die Anpassung des konstanten Faktors auf $6.95$ soll also folgendes garantieren:
\newline
Ist $P_{ij} = 1$, so wird im randomisierten Teil ein eindeutiger Zeuge mit
Wahrscheinlichkeit $\geq 1 - n^{-2}$ gefunden.

\begin{proof}
  Wir bezeichnen die Anzahl der Zeugen f{\"u}r $P_{ij}$ mit $w$.

  Die {\"a}u{\ss}ere Schleife wird mindestens mit einem $r$ durchlaufen, sodass
  $\frac{n}{2} \leq wr \leq n$ gilt. Die Wahrscheinlichkeit, dass in diesem Lauf
  $R$ keinen eindeutigen Zeugen f{\"u}r $P_{ij}$ enth{\"a}lt, ist maximal

  \begin{equation}
    \left( 1 - \frac{1}{2e} \right)
  \end{equation}

  Die Wahrscheinlichkeit, dass kein eindeutiger Zeuge f{\"u}r $P_{ij}$ gefunden
  wird, ist damit (f{\"u}r $\left\lceil 6.95 log(n) \right\rceil$ L{\"a}ufe)
  maximal
  
  \begin{equation}
    \left( 1 - \frac{1}{2e} \right)^{\left\lceil 6.95 log(n) \right\rceil} \leq n^{-2}
  \end{equation}

  \clearpage

  Damit ergibt sich mit der Gegenwahrscheinlichkeit, dass ein eindeutiger Zeuge
  mit einer Wahrscheinlichkeit

  \begin{equation}
    1 - \left( 1 - \frac{1}{2e} \right)^{\left\lceil 6.95 log(n) \right\rceil} \geq 1 - n^{-2}
  \end{equation}

  gefunden wird.

  Dass $x = 6.95$ diese Ungleichung erf{\"u}llt, zeigen wir im Folgenden:

  Die folgende Ungleichung muss erf{\"u}llt sein, damit Gleichung (8) gilt:

  \begin{equation}
    \left( 1 - \frac{1}{2e} \right)^{x log(n)} \leq n^{-2}
  \end{equation}

  Aus Gleichung (9) kann nun eine Schranke f{\"u}r $x$ berechnet werden.

  \begin{equation}
    \left( 1 - \frac{1}{2e} \right)^{x log(n)} \leq n^{-2}
  \end{equation}

  \begin{equation}
    log \left( \left( 1 - \frac{1}{2e} \right)^{x log(n)} \right) \leq log \left( n^{-2} \right)
  \end{equation}

  \begin{equation}
    x \cdot log \left( n \right) \cdot log \left( 1 - \frac{1}{2e} \right) \leq (-2) \cdot log \left( n \right)
  \end{equation}

  \begin{equation}
    x \cdot log \left( 1 - \frac{1}{2e} \right) \leq -2
  \end{equation}

  Da $log \left( 1 - \frac{1}{2e} \right)$ eine negative Zahl ist, muss beim
  Aufl{\"o}sen der Ungleichung auch $\leq$ in $\geq$ ge{\"a}ndert werden.

  \begin{equation}
    x \geq \frac{-2}{log \left( 1 - \frac{1}{2e} \right)} = 6.82006
  \end{equation}

  Aus Gleichung (14) ergibt sich also, dass alle $x \geq 6.82006$ Gleichung (9)
  erf{\"u}llen. \newline

  Daraus folgt, dass Gleichung (8) f{\"u}r alle $x \leq 6.8175866$ g{\"u}ltig ist.

  Wir w{\"a}hlen nun $x = 6.95$, was Gleichung (14) erf{\"u}llt. \newline
  Damit garantiert der angepasste $BPWM$-Algorithmus (vgl. Beginn Aufgabe 7),
  eine Laufzeit von $O \left( MM \left( n \right) \cdot log^2(n) \right)$ mit
  Wahrscheinlichkeit $1 - n^{-2}$  (vgl. Gleichung (8)).

  Dass die Zeilen 14 - 18 (bzw. die Zeilen 14 - 16 vom originalen Code auf Folie
  23) entfernt werden k{\"o}nnen, ergibt sich, da nachdem der randomisierte
  Teil des Codes ausgef{\"u}hrt wurde, erwartet nurmehr

  \begin{equation}
    \frac{1}{n^2} \cdot n^2 = 1
  \end{equation}

  Zeugen fehlen.

  W{\"u}rde man $x$ nun noch wesentlich h{\"o}her w{\"a}hlen, dann w{\"u}rde
  nach der Ausf{\"u}hrung des randomisierten Teils des Codes erwartet nurmehr
  $< 1$ Zeugen fehlen.
\end{proof}

\newpage
{\noindent\bfseries Aufgabe 8}%
\medskip%

\noindent
Es sei $n \in \mathbb{N}$, $w \in \{ 1, \ldots, n \}$, $r \in \mathbb{N}$ mit
\begin{center}
  $\frac{n}{2 \cdot c} \leq w \cdot r \leq \frac{n}{c}$, mit $c \geq 2$.
\end{center}
Eine Urne enthalte $n$ B{\"a}lle, von denen $w$ wei{\ss} und $(n - w)$ schwarz sind.
Zieht man zuf{\"a}llig $r$ B{\"a}lle aus der Urne, ohne diese zwischendurch
zur{\"u}ckzulegen, so ist
\begin{center}
  Pr[ genau ein wei{\ss}er Ball wurde gezogen ] = ?
\end{center}

\begin{proof}
  Pr[ genau ein wei{\ss}er Ball wurde gezogen ]
    = $\frac{\binom{w}{1} \cdot \binom{n - w}{r - 1}}{\binom{n}{r}}$

    = $w \cdot \frac{(n - w)!}{(n - w - r + 1)! \cdot (r - 1)!} \cdot \frac{r! \cdot (n - r)!}{n!}$
    = $w \cdot r \cdot \frac{(n - w)!}{(n - w - r + 1)!} \cdot \frac{(n - r)!}{n!}$

    = $w \cdot r \cdot \left( \prod\limits_{i = 0}^{w - 1} \frac{1}{n - i} \right) \cdot \left( \prod\limits_{i = 0}^{w - 2} \left( \left( n - r \right) - i \right) \right)$
    = $w \cdot r \cdot \frac{1}{n} \left( \prod\limits_{i = 0}^{w - 2} \frac{n - r - i}{n - i - 1} \right)$

    $\geq \frac{w \cdot r}{n} \cdot \left( \prod\limits_{i = 0}^{w - 2} \frac{n - r - w + 1}{n - w} \right)$
    = $\frac{w \cdot r}{n} \cdot \left( \frac{\left( n - w \right) - \left( r - 1 \right)}{n - w} \right)^{w - 1}$
    = $\frac{w \cdot r}{n} \left( 1 - \frac{r - 1}{n - w} \right)^{w - 1}$
   
  \bigskip
  \noindent
  Weiters wissen wir, dass
  \begin{enumerate}
    \item[i.)] $\frac{r - 1}{n - w} \leq \frac{1}{w}
        \Longleftrightarrow r \cdot  - w \leq n - w
        \Longleftrightarrow r \cdot w \leq n \text{ (per Lemma)}
      $
    \item[ii.)] $\frac{n}{2 \cdot c} \leq w \cdot r \leq \frac{n}{c}
        \Longleftrightarrow \frac{1}{2 \cdot c} \leq \frac{w \cdot r}{n} \leq \frac{1}{c}
      $
    \item[iii.)] $\left( 1 - \frac{1}{k} \right)^k \rightarrow \frac{1}{c}$ und
      $\left( 1 - \frac{1}{k} \right)^{k - 1} \geq \frac{1}{e}$
  \end{enumerate}

  \bigskip
  \noindent
  Setzt man nun i.) - iii.) in obige Formel ein, dann folgt:

  $\geq \frac{1}{2 \cdot c} \cdot \left( 1 - \frac{1}{w} \right)^{w - 1}$
  $\geq \frac{1}{2 \cdot c} \cdot \frac{1}{e} = \frac{1}{2 \cdot c \cdot e},
    \qquad \forall c \geq 2
  $
\end{proof}

\noindent
Das hei{\ss}t also, je gr{\"o}{\ss}er $c$ gew{\"a}hlt wird, desto kleiner wird die
Wahrscheinlichkeit genau einen wei{\ss}en Ball aus den $n$ B{\"a}llen zu ziehen.

\end{document}
