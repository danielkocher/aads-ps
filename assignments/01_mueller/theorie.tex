\documentclass{article}
\usepackage{graphicx}
\usepackage[utf8]{inputenc}
\usepackage[]{algorithm2e}
\begin{document}

\title{Advanced Algorithms and Datastructures (Assignment 1)}
\author{Christian Müller, Daniel Kocher}
\maketitle
\newpage
\begin{algorithm}[H]

\SetKwProg{Fn}{Function}{}{}
\SetKwInOut{Input}{input}
\SetKwInOut{Output}{output}

 \Input{Menge von horizontalen Segmenten $H$  und eine Menge von vertikalen Segmenten $V$. Sei $H^{\ast}$ die Menge der Endpunkte der horizontalen Segmente und $S=H^{\ast} \cup V$.}
 \Output{Paare $(h,v)$ mit $h \in H$ und $v \in V$.}
Sortiere Element von S gemäß ihrer horizontalen Position.\;
ReportCuts($S$,$L_{S}$,$R_{S}$,$V_{S}$)\;
\Fn{ReportCuts ($S$,$L_{S}$,$R_{S}$,$V_{S}$)}{

 \eIf{$|S|==1$ \tcp*{Sei p dieses Element}}{
   \Begin{
        \Switch{p}{
            \Case{p ist linker Endpunkt}{
                $L_{S}\leftarrow p_{y}$, $R_{S}\leftarrow \emptyset$, $V_{S}\leftarrow \emptyset$ \Return;
            }
             \Case{p ist rechter Endpunkt}{
                $L_{S}\leftarrow \emptyset$, $R_{S}\leftarrow p_{y}$, $V_{S}\leftarrow \emptyset$ \Return;
            }
              \Case{p vertikales Segment}{
                $L_{S}\leftarrow \emptyset$, $R_{S}\leftarrow \emptyset$, $V_{S}\leftarrow (p_{y1},p_{y2})$ \Return;
            }
        }
    }
   
   }{
   Wähle eine geeignete vertikale Trennlinie (dh. beliebige x-Koordinate), welche die Menge S in zwei gleich große Teilmengen $S1$ und $S2$ teilt\;
    ReportCuts($S1$,$L_{S1}$,$R_{S1}$,$V_{S1}$)\;
    ReportCuts($S2$,$L_{S2}$,$R_{S2}$,$V_{S2}$)\;
    $tmp \leftarrow L_{S1} \cap R_{S2}$\;
    \tcc{$tmp$ enhält so alle horizontalen Segmente, die in $S1$ beginnen und in $S2$ enden.}
    print($(L_{S1} \setminus tmp) \odot V_{S2}$)\;
    \tcc{$(L_{S1} \setminus tmp)$ enhält  alle horizontalen Segmente, die in $S1$ beginnen und nicht in $S2$ enden. $h \odot V_{2}$ liefert alle $v \in V_{S2}$, die $h_{y}$ enthalten. Genaue Vorgehensweise ist den Folien der Vorlesung zu entnehmen.}
    print($(R_{S2} \setminus tmp) \odot V_{S1}$)\;
     \tcc{analog zu letztem print. Horizontale Segmente die in $S2$ enden und ihren Anfangspunkt nicht in $S1$ haben geschnitten mit den vertikalen Segmenten aus $S1$.}
    $L_{S}\leftarrow (L_{S1} \setminus tmp) \cup L_{S2}$\;
   	$R_{S}\leftarrow (R_{S2} \setminus tmp) \cup R_{S1}$\;
   	$V_{S}\leftarrow V_{S1} \cup V_{S2}$\;
   	\Return\;
  }
  
}
\caption{PLG}
 \caption{CutReporter}
\end{algorithm}

\end{document}