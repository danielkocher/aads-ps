\documentclass{article}

\usepackage{amsmath,amsthm,amssymb}
\usepackage{commath}
\usepackage{mathtools}

\usepackage[utf8]{inputenc}
\begin{document}
\title{Assignment 3 \\ Advanced Algorithms \& Data Structures PS}%
\author{Christian Müller 1123410 \\ Daniel Kocher, 0926293}%
\maketitle

{\noindent\bfseries Aufgabe 5}%
\medskip%
 
\begin{proof}
\noindent
Zu zeigen:
\begin{enumerate}
  \item[i.)] Ist $D_{ij}$ ungerade, so gilt $D^{'}_{kj} \leq D^{'}_{ij}$ für alle Nachbarn k von i und
  \item[ii.)] es gibt einen Knoten $k \in \Gamma(i)$, so dass $D^{'}_{kj} < D^{'}_{ij}$.
\end{enumerate}
\begin{equation}
  D_{ij}(\text{mod } 2) = 1
\end{equation}
\begin{equation}
  \implies \exists l \in \mathbb{N}: D_{ij} = 2l - 1
\end{equation}

\noindent
und es gilt (Lemma aus VO):
\begin{equation}
  D_{ij} = 2D^{'}_{ij} - 1 \text{(f{\"u}r $D_{ij}$ ungerade)}
\end{equation}
\begin{equation}
  D_{ij} = 2D^{'}_{ij} \text{(f{\"u}r $D_{ij}$ gerade)}
\end{equation}

\noindent
aus (2) und (3) folgt: 
\begin{equation}
  2l-1 = 2D^{'}_{ij}-1 \implies D^{'}_{ij} = l
\end{equation}

\noindent
Der Knoten k ist direkter Nachbar von i also gilt (Lemma aus VO):
\begin{equation}
  D_{ij} - 1 \leq D_{kj} \leq D_{ij} + 1
\end{equation}

\noindent
Fall 1: $D_{kj} = D_{ij} - 1$
\begin{equation}
  D_{kj} = D_{ij} - 1 = 2l - 2
\end{equation}
\begin{equation}
  D_{kj}(\text{mod } 2) = 0 \xRightarrow{(4)} D_{kj}=2D^{'}_{kj}
\end{equation}
\begin{equation}
  2l-2 = 2D^{'}_{kj} \implies D^{'}_{kj}=l-1
\end{equation}
\begin{equation}
  D^{'}_{ij} =l \wedge D^{'}_{kj} =l-1 \implies D^{'}_{kj} < D^{'}_{ij}
\end{equation}

\noindent
Fall 2: $D_{kj} = D_{ij}$
\begin{equation}
  D_{kj} = D_{ij}=2l-1
\end{equation}
\begin{equation}
  D_{kj}(\text{mod } 2) = 1 \xRightarrow{(4)} D_{kj} = 2D^{'}_{kj} - 1
\end{equation}
\begin{equation}
  2l - 1 = 2D^{'}_{kj} - 1 \implies D^{'}_{kj} = l
\end{equation}
\begin{equation}
  D^{'}_{ij} = l \wedge D^{'}_{kj} = l \implies D^{'}_{kj} \leq D^{'}_{ij}
\end{equation}

\noindent
Fall 3: $D_{kj} = D_{ij} + 1$
\begin{equation}
  D_{kj} = D_{ij} + 1 = 2l
\end{equation}
\begin{equation}
  D_{kj}(\text{mod } 2) = 0 \xRightarrow{(4)} D_{kj} = 2D^{'}_{kj}
\end{equation}
\begin{equation}
  2l = 2D^{'}_{kj} \implies D^{'}_{kj} = l
\end{equation}
\begin{equation}
  D^{'}_{ij} = l \wedge D^{'}_{kj} = l \implies D^{'}_{kj} \leq D^{'}_{ij}
\end{equation}

\noindent
Damit sind i.) und ii.) wie folgt bewiesen:
\begin{itemize}
  \item[] i.) durch (10), (14) und (18).
  \item[] ii.) durch (10) und der Tatsache, dass es einen Knoten $k$ mit $D_{kj} = D_{ij} - 1$ geben muss. Der Knoten $k$ ist der erste Knoten der auf dem kürzesten Weg von $i$ nach $j$ besucht wird. Würde ein solcher Knoten nicht existieren gäbe es auch diesen Weg nicht.
\end{itemize}
\end{proof}

\newpage
{\noindent\bfseries Aufgabe 6}%
\medskip%

\begin{proof}
\noindent
Zu zeigen:
\begin{enumerate}
  \item[i.)] $D_{ij}$ ist gerade genau dann, wenn $\sum_{k \in \Gamma(i)} D^{'}_{kj} \geq D^{'}_{ij} \cdot \text{deg}(i)$.
  \item[ii.)] $D_{ij}$ ist ungerade genau dann, wenn $\sum_{k \in \Gamma(i)} D^{'}_{kj} < D^{'}_{ij} \cdot \text{deg}(i)$.
\end{enumerate}

\begin{equation}
  D_{ij}(\text{mod } 2) = 0
\end{equation}
\begin{equation}
  \implies \exists l \in \mathbb{N}: D_{ij} = 2l
\end{equation}

\noindent
und es gilt:
\begin{equation}
  D_{ij} = 2D^{'}_{ij}
\end{equation}

\noindent
aus (20) und (21) folgt: 
\begin{equation}
  2l = 2D^{'}_{ij} \implies D^{'}_{ij} = l
\end{equation}

\noindent
Der Knoten k ist direkter Nachbar von i also gilt:
\begin{equation}
  D_{ij} - 1 \leq D_{kj} \leq D_{ij} + 1
\end{equation}

\noindent
Fall 1: $D_{kj} = D_{ij} - 1$
\begin{equation}
  D_{kj} = D_{ij} - 1  = 2l - 1
\end{equation}
\begin{equation}
  D_{kj}(\text{mod } 2) = 1 \implies D_{kj} = 2D^{'}_{kj} - 1
\end{equation}
\begin{equation}
2l - 1 = 2D^{'}_{kj} - 1 \implies D^{'}_{kj} = l
\end{equation}

\noindent
Fall 2: $D_{kj} = D_{ij}$
\begin{equation}
  D_{kj} = D_{ij} = 2l
\end{equation}
\begin{equation}
  D_{kj}(\text{mod } 2) = 0 \implies D_{kj} = 2D^{'}_{kj}
\end{equation}
\begin{equation}
 2l = 2D^{'}_{kj} \implies D^{'}_{kj} = l
\end{equation}

\noindent
Fall 3: $D_{kj} = D_{ij} + 1$
\begin{equation}
  D_{kj} = D_{ij} + 1 = 2l + 1
\end{equation}
\begin{equation}
  D_{kj}(\text{mod } 2) = 1 \implies D_{kj} = 2D^{'}_{kj} - 1
\end{equation}
\begin{equation}
  2l + 1 = 2D^{'}_{kj} - 1 \implies D^{'}_{kj} = l + 1
\end{equation}

\noindent
Es folgt also aus (26), (29) und (32):
\begin{equation}
  \forall k \in \Gamma(i): D^{'}_{kj} \geq l
\end{equation}

\noindent
Weiters folgt aus (10), (14) und (18) sowie ii.) (Aufgabe 5):
\begin{equation}
  \forall k \in \Gamma(i): D^{'}_{kj} < l
\end{equation}

\noindent
Sei $m=\abs{\Gamma(i)} = \text{deg}(i)$, dann folgen daraus die sich gegenseitig ausschlie{\ss}enden Formeln:
\begin{equation}
  \sum_{k \in \Gamma(i)} D^{'}_{kj} \geq l \cdot m = D^{'}_{ij} \cdot \text{deg}(i)
\end{equation}
\begin{equation}
  \sum_{k \in \Gamma(i)} D^{'}_{kj} < l \cdot m = D^{'}_{ij} \cdot \text{deg}(i)
\end{equation}

\noindent
Aus (35) und (36) folgt dann das Lemma der Angabe.
\end{proof}

\end{document}
