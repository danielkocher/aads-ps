\documentclass{article}

\usepackage{amsmath,amsthm,amssymb}
\usepackage{commath}

\usepackage[utf8]{inputenc}
\begin{document}
\title{Assignment 3 \\ Advanced Algorithms \& Data Structures PS}%
\author{Christian Müller 1123410 \\ Daniel Kocher, 0926293}%
\maketitle


 
 \begin{proof}
 Zu zeigen: Ist $D_{ij}$ ungerade so ist $D^{'}_{kj} \leq D^{'}_{ij}$ für alle Nachbarn k von i.
\begin{equation}
D_{ij}(mod 2) = 1
\end{equation}
\begin{equation}
\implies \exists l \in N: D_{ij} = 2l-1
\end{equation}
und es gilt:
\begin{equation}
D_{ij} = 2D^{'}_{ij}-1
\end{equation}
aus (2) und (3) folgt: 
\begin{equation}
2l-1 = 2D^{'}_{ij}-1 \implies D^{'}_{ij} = l
\end{equation}
Der Knoten k ist direkter Nachbar von i also gilt:
\begin{equation}
D_{ij}-1 \leq D_{kj} \leq D_{ij}+1
\end{equation}

Fall 1: $D_{kj} = D_{ij}-1$
\begin{equation}
D_{kj} = D_{ij}-1=2l-2
\end{equation}
\begin{equation}
D_{kj}(mod2) = 0 \implies D_{kj}=2D^{'}_{kj}
\end{equation}
\begin{equation}
2l-2 = 2D^{'}_{kj} \implies D^{'}_{kj}=l-1
\end{equation}
\begin{equation}
D^{'}_{ij} =l \wedge D^{'}_{kj} =l-1 \implies D^{'}_{kj} \leq D^{'}_{ij}
\end{equation}

Fall 2: $D_{kj} = D_{ij}$
\begin{equation}
D_{kj} = D_{ij}=2l-1
\end{equation}
\begin{equation}
D_{kj}(mod2) = 1 \implies D_{kj}=2D^{'}_{kj}-1
\end{equation}
\begin{equation}
2l-1 = 2D^{'}_{kj}-1 \implies D^{'}_{kj}=l
\end{equation}
\begin{equation}
D^{'}_{ij} =l \wedge D^{'}_{kj} =l \implies D^{'}_{kj} \leq D^{'}_{ij}
\end{equation}

Fall 3: $D_{kj} = D_{ij}+1$
\begin{equation}
D_{kj} = D_{ij}+1=2l
\end{equation}
\begin{equation}
D_{kj}(mod2) = 0 \implies D_{kj}=2D^{'}_{kj}
\end{equation}
\begin{equation}
2l = 2D^{'}_{kj} \implies D^{'}_{kj}=l
\end{equation}
\begin{equation}
D^{'}_{ij} =l \wedge D^{'}_{kj} =l \implies D^{'}_{kj} \leq D^{'}_{ij}
\end{equation}

\end{proof}
\newpage
\begin{proof}
 Zu zeigen: $D_{ij}$ ist gerade genau dann, wenn $\sum_{k \in \Gamma(i)} D^{'}_{kj} \geq D^{'}_{ij} * deg(i)$.

\begin{equation}
D_{ij}(mod 2) = 0
\end{equation}
\begin{equation}
\implies \exists l \in N: D_{ij} = 2l
\end{equation}
und es gilt:
\begin{equation}
D_{ij} = 2D^{'}_{ij}
\end{equation}
aus (2) und (3) folgt: 
\begin{equation}
2l = 2D^{'}_{ij} \implies D^{'}_{ij} = l
\end{equation}
Der Knoten k ist direkter Nachbar von i also gilt:
\begin{equation}
D_{ij}-1 \leq D_{kj} \leq D_{ij}+1
\end{equation}

Fall 1: $D_{kj} = D_{ij}-1$
\begin{equation}
D_{kj} = D_{ij}-1=2l-1
\end{equation}
\begin{equation}
D_{kj}(mod2) = 1 \implies D_{kj}=2D^{'}_{kj}-1
\end{equation}
\begin{equation}
2l-1 = 2D^{'}_{kj}-1 \implies D^{'}_{kj}=l
\end{equation}

Fall 2: $D_{kj} = D_{ij}$
\begin{equation}
D_{kj} = D_{ij}=2l
\end{equation}
\begin{equation}
D_{kj}(mod2) = 0 \implies D_{kj}=2D^{'}_{kj}
\end{equation}
\begin{equation}
2l = 2D^{'}_{kj} \implies D^{'}_{kj}=l
\end{equation}

Fall 3: $D_{kj} = D_{ij}+1$
\begin{equation}
D_{kj} = D_{ij}+1=2l+1
\end{equation}
\begin{equation}
D_{kj}(mod2) = 1 \implies D_{kj}=2D^{'}_{kj}-1
\end{equation}
\begin{equation}
2l+1 = 2D^{'}_{kj}-1 \implies D^{'}_{kj}=l+1
\end{equation}
Es gilt also:
\begin{equation}
\forall k \in \Gamma(i): D^{'}_{kj} \geq l
\end{equation}
Sei $m=\abs{\Gamma(i)}=deg(i)$, dann:
\begin{equation}
\sum_{k \in \Gamma(i)} D^{'}_{kj} \geq m*l = D^{'}_{ij} * deg(i)  
\end{equation}
\end{proof}
\end{document}
